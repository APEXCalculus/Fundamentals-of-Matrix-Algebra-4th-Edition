\section{Cramer's Rule}\label{sec:cramer}

\asyouread{
\item	T/F: Cramer's Rule is another method to compute the determinant of a matrix.
\item	T/F: Cramer's Rule is often used because it is more efficient than Gaussian elimination.
\item Mathematicians use what word to describe the connections between seemingly unrelated ideas?
}


In the previous sections we have learned about the determinant, but we haven't given a really good reason \textit{why} we would want to compute it.\footnote{The closest we came to motivation is that if $\det{\tta} =0$, then we know that \tta\ is not invertible. But it seems that there may be easier ways to check.} This section shows one application of the determinant: solving systems of linear equations. We introduce this idea in terms of a theorem; then we will practice.

\theorem{thm:cramers_rule}{\index{Cramer's Rule}\textbf{Cramer's Rule}\\

Let \tta\ be an $n\times n$ matrix with $\det{\tta}\neq 0$ and let \vb\ be an $n\times 1$ column vector. Then the linear system $$\ttaxb$$ has solution $$x_i = \frac{\det{\tta_i(\vb)}}{\det{\tta}},$$ where $\tta_i(\vb)$ is the matrix formed by replacing the $i^\text{th}$ column of \tta\ with \vb.}

Let's do an example.\\

\example{ex_cramer_1}{Use Cramer's Rule to solve the linear system \ttaxb\ where $$\tta = \bmx{ccc}1&5&-3\\1&4&2\\2&-1&0 \emx \ \text{ and }\ \vb = \bmx{c}-36\\-11\\7\emx.$$}
{We first compute the determinant of \tta\ to see if we can apply Cramer's Rule. $$\det{\tta} = \bdt{ccc}1&5&-3\\1&4&2\\2&-1&0\edt = 49.$$

Since $\det{\tta}\neq 0$, we can apply Cramer's Rule. Following Theorem \ref{thm:cramers_rule}, we compute \det{\tta_1(\vb)}, \det{\tta_2(\vb)} and \det{\tta_3(\vb)}. 

$$\det{\tta_1(\vb)} = \bdt{ccc} {\bf -36}&5&-3\\{\bf -11}&4&2\\{\bf 7}&-1&0 \edt = 49.$$
(We used a bold font to show where \vb\ replaced the first column of \tta.)

$$\det{\tta_2(\vb)} = \bdt{ccc} 1&{\bf -36}&-3\\1&{\bf -11}&2\\2&{\bf 7}&0\edt = -245.$$

$$\det{\tta_3(\vb)} = \bdt{ccc}  1&5&{\bf -36}\\1&4&{\bf -11}\\2&-1&{\bf 7}\edt = 196.$$

Therefore we can compute \vx:
\begin{align*}
x_1 &= \frac{\det{\tta_1(\vb)}}{\det{\tta}} = \frac{49}{49} = 1\\
x_2 &= \frac{\det{\tta_2(\vb)}}{\det{\tta}} = \frac{-245}{49} = -5\\
x_3 &= \frac{\det{\tta_3(\vb)}}{\det{\tta}} = \frac{196}{49} = 4
\end{align*}

Therefore $$\vx = \bmx{c}x_1\\x_2\\x_3\emx = \bmx{c}1\\-5\\4\emx.$$
\ } \\ %\eexset

Let's do another example.\\

\example{ex_cramer_2}{Use Cramer's Rule to solve the linear system \ttaxb\ where $$\tta = \bmx{cc} 1&2\\3&4\emx \ \text{ and } \ \vb = \bmx{c} -1\\1\emx.$$}
{The determinant of \tta\ is $-2$, so we can apply Cramer's Rule. 
$$\det{\tta_1(\vb)} = \bdt{cc} {\bf -1} & 2\\ {\bf 1} & 4\edt = -6.$$

$$\det{\tta_2(\vb)} = \bdt{cc} 1 & {\bf -1}\\ 3 & {\bf 1} \edt = 4.$$

Therefore 
\begin{align*}
x_1 &= \frac{\det{\tta_1(\vb)}}{\det{\tta}} = \frac{-6}{-2} = 3\\
x_2 &= \frac{\det{\tta_2(\vb)}}{\det{\tta}} = \frac{4}{-2} = -2\\
\end{align*}
and $$\vx = \bmx{c}x_1\\x_2\emx = \bmx{c}3\\-2\emx.$$ \ } \\ %\eexset

We learned in Section \ref{sec:determinant_properties} that when considering a linear system \ttaxb\ where \tta\ is square, if $\det{\tta}\neq 0$ then \tta\ is invertible and \ttaxb\ has exactly one solution. We also stated in Key Idea \ref{idea:solutions_invert} that if $\det{\tta} = 0$, then \tta\ is not invertible and so therefore either \ttaxb\ has no solution or infinite solutions. Our method of figuring out which of these cases applied was to form the augmented matrix $\bmx{cc} \tta & \vb \emx$, put it into \rref, and then interpret the results.

Cramer's Rule specifies that $\det{\tta}\neq 0$ (so we are guaranteed a solution). % New material for 3rd Edition
When $\det{\tta}=0$ we are not able to discern whether infinite solutions or no solution exists for a given vector \vb. Cramer's Rule is only applicable to the case when exactly one solution exists. \\

%What if $\det{\tta}=0$? Can we determine anything using Cramer's Rule--like techniques? The answer is yes, which we'll demonstrate through the next example.\\
%
%\example{ex_cramer_3}{Analyze the linear systems \ttaxb\ using Cramer's Rule, where $$\tta\ = \bmx{ccc} 1&2&3\\4&5&6\\7&8&9\emx, \ \text{ and with } \ \vb = \bmx{c}1\\1\\1\emx \ \text{ and }\ \bmx{c} 1\\1\\0\emx.$$}
%{We first compute \det{\tta}, and find that $\det{\tta}=0$, so we can't apply Cramer's Rule. That is, if the determinant were not 0, we would know that we have exactly one solution and a method of finding it. Since the determinant is 0, we simply \textit{don't know} if a solution exists; Cramer's Rule \textbf{does not} say that the solution does not exist.
%
%Let's try to analyze the system \ttaxb\ where $\vb = \bmx{c}1\\1\\1\emx$. Using Cramer's Rule--type concepts, we compute
%$$\det{\tta_1(\vb)} = \bdt{ccc}{\bf 1}&2&3\\{\bf 1}&5&6\\{\bf 1}&8&9\edt = 0.$$
%
%Similar computations show that \det{\tta_2(\vb)}\ and \det{\tta_3(\vb)}\ are all 0. What does this mean? We don't know yet, but let's try to solve \ttaxb\ by putting into \rref\ the augmented matrix $\bmx{cc} \tta & \vb\emx$ and looking at the result.
%$$\bmx{cccc}1&2&3&1\\4&5&6&1\\7&8&9&1\emx \quad\quad \overrightarrow{\text{ rref }}\quad\quad \bmx{cccc}1&0&-1&-1\\0&1&2&1\\0&0&0&0\emx$$
%
%This shows us that we have infinite solutions to the equation \ttaxb.
%
%Now, consider \ttaxb\ again, this time with $\vb = \bmx{c}1\\1\\0\emx$. Applying Cramer's Rule--like ideas, 
%$$\det{\tta_1(\vb)} = \bdt{ccc}{\bf 1}&2&3\\{\bf 1}&5&6\\{\bf 0}&8&9\edt = 3.$$
%
%Similar computations show that $\det{\tta_2(\vb)}=-6$ and $\det{\tta_3(\vb)}=3$. Finding the solution using an augmented matrix, we find that 
%$$\bmx{cccc}1&2&3&1\\4&5&6&1\\7&8&9&0\emx \quad\quad \overrightarrow{\text{ rref }}\quad\quad \bmx{cccc}1&0&-1&0\\0&1&2&0\\0&0&0&1\emx$$
%This shows that there is no solution.\\ } %\eexset
%
%We ended the above example without a conclusion for we offer it here. Here are the key ideas: we had two linear systems of the form \ttaxb\ where $\det{\tta}=0$. For one vector \vb, we had $\det{\tta_i(\vb)}=0$ for all $i$ and infinite solutions existed; for the other vector \vb, we had $\det{\tta_i(\vb)}\neq 0$ for all $i$, and there was no solution. 
%
%The determinants of the matrices $\tta_i(\vb)$ are the key, which we explicitly state next.
%
%\keyidea{idea:cramers}{\index{Cramer's Rule!and noninvertible matrices}\textbf{Cramer's Rule for Noninvertible Matrices}\\
%
%Let \tta\ be an $n\times n$ matrix with $\det{\tta}=0$, let \vb\ be an $n\times 1$ column vector, and consider the linear system \ttaxb. If $\det{\tta_i(\vb)}=0$ for all $i$, then the system \ttaxb\ has infinite solutions; otherwise, it has no solution.}
%
%A key point of the Key Idea is that if $\det{\tta_i(\vb)}\neq 0$ for even one case, then the system is inconsistent. 

We end this section with a practical consideration. We have mentioned before that finding determinants is a computationally intensive operation. To solve a linear system with 3 equations and 3 unknowns, we need to compute 4 determinants. Just think: with 10 equations and 10 unknowns, we'd need to compute 11 really hard determinants of $10\times 10$ matrices! That is a lot of work!

The upshot of this is that Cramer's Rule makes for a poor choice in solving numerical linear systems. It simply is not done in practice; it is hard to beat Gaussian elimination.\footnote{A version of Cramer's Rule is often taught in introductory differential equations courses as it can be used to find solutions to certain linear differential equations. In this situation, the entries of the matrices are functions, not numbers, and hence computing determinants is easier than using Gaussian elimination. Again, though, as the matrices get large, other solution methods are resorted to.}

So why include it? \textit{Because its truth is amazing.} The determinant is a very strange operation; it produces a number in a very odd way. It should seem incredible to the reader that by manipulating determinants in a particular way, we can solve linear systems.

%It is not accurate to say that Cramer's Rule is \textit{never} used. For certain elementary linear differential equations, especially at the introductory level, Cramer's Rule provides a systematic way of arriving at a solution. Again, however, there are generally ``better'' ways of finding the solution.

In the next chapter we'll see another use for the determinant. Meanwhile, try to develop a deeper appreciation of math: odd, complicated things that seem completely unrelated often are intricately tied together. Mathematicians see these connections and describe them as ``beautiful.''\\

\printexercises{exercises/03_05_exercises}

%\Large \textbf{Answers to Odd Numbered Exercises}

%\begin{multicols}{2}
%\printanswers{exercises/03_05_exercises}
%\end{multicols}
