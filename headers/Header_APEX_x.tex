\documentclass[10pt]{book}

\usepackage{ifthen}

%%%%
%% Begins the usepackage section
%%%%

\usepackage{amsmath}
\usepackage{amssymb}
\usepackage{amsthm}


%%
% slight detour to create the proper text/paper size.
% Depends on the ifthen package, needed before loading
% the geometry package
%%
\newboolean{sevenbytenbooksize}
\newboolean{ipadsize}   % for the ipad, Duh.

\setboolean{sevenbytenbooksize}{true}
\setboolean{ipadsize}{false}
%
%	Note how the iPad version differs from the booksize only in 
% the size of the margins.
%
% Changing the following isn't hard but should be done with 
% practice and care, making sure to get the margins right for a 
% particular page size and printing format
%
%% Layout for printed book through Amazon CreateSpace, perfect bound,
%% 7x10 ``journal'' size

\ifthenelse{\boolean{ipadsize}}{\usepackage[paperheight=620pt,paperwidth=395pt, textheight=550pt,
textwidth=345pt, includeheadfoot, hcentering=true]{geometry}}{}

\ifthenelse{\boolean{sevenbytenbooksize}}{\usepackage[paperheight=10in,paperwidth=7in, textheight=550pt,
textwidth=345pt, inner=1in,includeheadfoot]{geometry}}{}

%% end detour
%%
\usepackage{layout}
\usepackage{graphicx}
\usepackage{multicol}
\usepackage{tikz}
\usetikzlibrary{calc}
\usepackage{makeidx}

\usepackage{changepage}
\usepackage{ulem}
\usepackage{xcolor}
\usepackage[bookmarksnumbered,linkbordercolor=white]{hyperref}

\usepackage{fancyhdr}
\usepackage{calc}

\makeindex

%%%%
%% Commands for the header, utilizing the fancyhdr
%% (fancy header) package
%%%%

\pagestyle{fancy}
\fancyhead{}
\fancyfoot{}
\renewcommand{\chaptermark}[1]{\markboth{\chaptername\ \thechapter\ \ \ \ {#1}}{}}
\renewcommand{\sectionmark}[1]{\markright{\thesection\ \ \ \  #1}}
\fancyhf{}         %Clears all header and footer fields, in preparation.

\fancyhfoffset[LE,RO]{30pt}
\fancyfoot[LE,RO]{\textbf{\thepage}} %Displays the page number in bold in the header,
                       % to the left on even pages and to the right on odd pages.        
\fancyhead[LE]{\nouppercase{\leftmark}}
      %Displays the upper-level (chapter) information---
      % as determined above---in non-upper case in the header, 
      %to the right on even pages.
\fancyhead[RO]{\rightmark}
			%Displays the lower-level (section) information---as
      % determined above---in the header, to the left on odd pages.
\renewcommand{\headrulewidth}{0pt}
\renewcommand{\footrulewidth}{0pt}
	%Underlines the header and footer. (Set to 0pt if not required).


%%%%
%% Commands to determine whether we print in color or 
%% black and white
%%%%
\newboolean{in_color}

% determines the line colors for color and black and white lines.
\newcommand{\colorlinecolor}{blue!95!black!30}
\newcommand{\bwlinecolor}{black!30}

% sets the line color to be in color, as a default
\newcommand{\thelinecolor}{\colorlinecolor}

% this allows the above default to be overriden by using
% the \printincolor and \printinblackandwhite commands
% anywhere in the file. This allows you to switch back
% and forth between bw and color. (Who would want to?)
\newcommand{\printincolor}{\setboolean{in_color}{true}%
\renewcommand{\thelinecolor}{\colorlinecolor}}
\newcommand{\printinblackandwhite}{\setboolean{in_color}{false}%
\renewcommand{\thelinecolor}{\bwlinecolor}}

% the default is printing in color
\printincolor

%% Creates a lot of measurements - lengths - to use
%% later on. Explained when a value is set.
\newlength{\topmarginlength} 
\newlength{\bottommarginlength}
\newlength{\oddpagemarginlength}
\newlength{\evenpagemarginlength}
\newlength{\marginlinelength}


% measures how far from the text the example line is to be drawn
\setlength{\marginlinelength}{5pt}

% the height of the top margin
% used in calculating the lines for examples
\setlength{\topmarginlength}{-1in-\voffset}

% the length of the bottom margin (ish)
% actually starts at the top of the page, moves
% through the top margin length then the text height.
\setlength{\bottommarginlength}{-1in-\textheight-2\baselineskip}

% the length of the left hand margin of an odd page
\setlength{\oddpagemarginlength}{1in+\hoffset+\oddsidemargin-2\marginlinelength}

% the length of the left hand margin of an even page
\setlength{\evenpagemarginlength}{1in+\hoffset+\evensidemargin-2\marginlinelength}


% creates a generic style for the lines. You can add lots
% of things here, all separated by commas.
\newcommand{\linestyle}{thick}

% Do you want to draw the lines for examples? If so,
% use the first command. Otherwise, use the second. 
% The first is the default, but you can override it
% by using the second in your main file.
\newboolean{showexamplelines}
\newcommand{\drawexamplelines}{\setboolean{showexamplelines}{true}}
\newcommand{\nodrawexamplelines}{\setboolean{showexamplelines}{false}}

% by default the lines around the examples are drawn
\drawexamplelines


% This is more a debugging tool than a stylistic one.
% This draws a small circle in the margin at the the
% beginning of an example and another immediately follwing
% the end of the example, in the text. Sometimes it is useful
% to figure out why there seems to be a lot of space between
% the end of the example and the end of the drawn line.
% Too much space can occur, for instance, if the example ends with
% with a ``$$ maths $$'' type environment. You can correct for this
% with a \vskip -\baselineskip command.
\newboolean{showexamplemarks}
\newcommand{\drawexamplemarks}{\setboolean{showexamplemarks}{true}}
\newcommand{\nodrawexamplemarks}{\setboolean{showexamplemarks}{false}}
\drawexamplemarks

\newcounter{examplecounter}
\setcounter{examplecounter}{0}

% the example environment has a regular and starred version.
% the regular version takes 3 arguments: label, problem, solution.
% The starred version only has 2: label, problem/solution.
\makeatletter
\newcommand{\example}{\@ifstar \examplestarred \examplenostar}
\makeatother

%% This is the no-star (regular) version of
%% the example command.
\newcommand{\examplenostar}[3]{%
%\ifthenelse{\boolean{in_color}}{renewcommand{\thelinecolor}{\colorlinecolor}}{\renewcommand{\thelinecolor}{\bwlinecolor}}%
\noindent%
\hskip -2\marginlinelength%
\parbox{\marginlinelength}{%
\begin{tikzpicture}[remember picture,overlay]%
\ifthenelse{\boolean{showexamplelines}}{\ifthenelse{\boolean{showexamplemarks}}
{\draw (0,0) circle (1pt) node (#1) {};}%
{\draw (0,0) node (#1) {};}
}
{}% ends the ``Draw example lines and marks?'' booleans
\end{tikzpicture}%
}% ends parbox where line begins
\hskip \marginlinelength% puts us back in line
\parbox{80pt}{{\bf Example \refstepcounter{examplecounter}\theexamplecounter}}% ends parbox
\label{#1}%
\ifthenelse{\pageref{#1}=\pageref{e#1}}% if the beginning and end are on the same page
{%
#2 \vskip \baselineskip%
\parbox{65pt}{\textsc{\small\bfseries Solution}}% 
#3  \label{e#1}% writes the full example then draws the lines
\ifthenelse{\boolean{showexamplelines}}{%
\begin{tikzpicture}[remember picture,overlay]%
\ifthenelse{\boolean{showexamplemarks}}{\draw (0,0) circle (1pt) node (e#1) {};}%
{\draw (0,0) node (e#1) {};}%
\draw [\linestyle,\thelinecolor] (#1) -- (#1.south |- e#1.south) -- ++(10pt,0);
\end{tikzpicture}
}{}% ends if/then/else show lines
}% ends if beginning and end are on same page.
% next is if they are on different pages
{% first draw line from start to bottom of page
\ifthenelse{\boolean{showexamplelines}}{%
\begin{tikzpicture}[remember picture,overlay]% 
\ifthenelse{\isodd{\pageref{#1}}}% draws lines based on whether on an even or odd page %
{\node [xshift=\oddpagemarginlength,yshift=\bottommarginlength](bottomleft) at (current page.north west)  {};}
{\node [xshift=\evenpagemarginlength,yshift=\bottommarginlength](bottomleft) at (current page.north west)  {};}
\draw [\thelinecolor,\linestyle] (#1) -- (bottomleft);
\end{tikzpicture}
}{}% ends if/then/else show lines
% end drawing of line
#2 \vskip \baselineskip%
\parbox{65pt}{\textsc{\small \bfseries Solution}}%
#3  \label{e#1}% now write out full example
% now draw line from end to top of page
\ifthenelse{\boolean{showexamplelines}}{%
\begin{tikzpicture}[remember picture,overlay] 
\ifthenelse{\boolean{showexamplemarks}}
{\draw (0,0) circle (1pt) node (e#1) {};}%
{\draw (0,0) node (e#1) {};}%
\ifthenelse{\isodd{\pageref{e#1}}}% draws lines based on whether on an even or odd page %
{\node [xshift=\oddpagemarginlength,yshift=\topmarginlength](topleft) at (current page.north west)  {};}
{\node [xshift=\evenpagemarginlength,yshift=\topmarginlength](topleft) at (current page.north west)  {};}
\draw [\thelinecolor,\linestyle] (topleft)--(e#1.south -| topleft) -- ++(10pt,0);
\end{tikzpicture}%
}{}% ends if/then/else show lines
}% ends the check for same page or not
%
}%ends the definition of example

% This is the starred version of example.
% It does not include a ``solution'' and only
% takes 2 arguments.

\newcommand{\examplestarred}[2]{%
\noindent%
\hskip -2\marginlinelength%
\parbox{\marginlinelength}{%
\begin{tikzpicture}[remember picture,overlay]%
\ifthenelse{\boolean{showexamplelines}}{\ifthenelse{\boolean{showexamplemarks}}
{\draw (0,0) circle (1pt) node (#1) {};}%
{\draw (0,0) node (#1) {};}
}
{}% ends the ``Draw example lines and marks?'' booleans
\end{tikzpicture}%
}% ends parbox where line begins
\hskip \marginlinelength% puts us back in line
\parbox{80pt}{{\bf Example \refstepcounter{examplecounter}\theexamplecounter}}% ends parbox
\label{#1}%
\ifthenelse{\pageref{#1}=\pageref{e#1}}% if the beginning and end are on the same page
{%
#2 \label{e#1}% writes the full example then draws the lines
\ifthenelse{\boolean{showexamplelines}}{%
\begin{tikzpicture}[remember picture,overlay] 
\ifthenelse{\boolean{showexamplemarks}}
{\draw (0,0) circle (1pt) node (e#1) {};}%
{\draw (0,0) node (e#1) {};}%
\draw [\linestyle,\thelinecolor] (#1) -- (#1.south |- e#1.south) -- ++(10pt,0);
\end{tikzpicture}
}{}% ends if/then/else show lines
}% ends if beginning and end are on same page.
% next is if they are on different pages
{% first draw line from start to bottom of page
\ifthenelse{\boolean{showexamplelines}}{%
\begin{tikzpicture}[remember picture,overlay]% 
\ifthenelse{\isodd{\pageref{#1}}}% draws lines based on whether on an even or odd page %
{\node [xshift=\oddpagemarginlength,yshift=\bottommarginlength](bottomleft) at (current page.north west)  {};}
{\node [xshift=\evenpagemarginlength,yshift=\bottommarginlength](bottomleft) at (current page.north west)  {};}
\draw [\thelinecolor,\linestyle] (#1) -- (bottomleft);
\end{tikzpicture}
}{}% ends if/then/else show lines
% end drawing of line
#2 \label{e#1}% now write out full example
% now draw line from end to top of page
\ifthenelse{\boolean{showexamplelines}}{%
\begin{tikzpicture}[remember picture,overlay] 
\ifthenelse{\boolean{showexamplemarks}}{\draw (0,0) circle (1pt) node (e#1) {};}%
{\draw (0,0) node (e#1) {};}%
\ifthenelse{\isodd{\pageref{e#1}}}% draws lines based on whether on an even or odd page %
{\node [xshift=\oddpagemarginlength,yshift=\topmarginlength](topleft) at (current page.north west)  {};}
{\node [xshift=\evenpagemarginlength,yshift=\topmarginlength](topleft) at (current page.north west)  {};}
\draw [\thelinecolor,\linestyle] (topleft)--(e#1.south -| topleft) -- ++(10pt,0);
\end{tikzpicture}%
}{}% ends if/then/else show lines
}% ends the check for same page or not
%
}%ends the definition of example


% Draws a line on a page that doesn't contain either
% the beginning or end of an example.
% Takes no arguments; figures out if you 
% are on an even or odd numbered page for
% correct margin calculation
\newcommand{\drawexampleline}{%
\begin{tikzpicture}[remember picture,overlay]
				\ifthenelse{\isodd{\thepage}}{%
				\node [xshift=\oddpagemarginlength,yshift=\topmarginlength](tleft) at (current page.north west)  {};}
				{\node [xshift=\evenpagemarginlength,yshift=\topmarginlength](tleft) at (current page.north west)  {};}
        \draw [\thelinecolor,thick] (tleft) -- ++(0,-\textheight);
        \end{tikzpicture}
}


%
%Define style for Definitions, Theorems and Key Ideas  %%%%%%%%%%%%%%%%%%%%%%%%%%%%%%%%%%%%%%%%%%%%
%%%%%%%%%%%%%%%%%%%%%%%%%%%%%%%%%%%%%%%%%%%%%%%%%%%%%%%%%%%%%%%%%%%%%%%%%%%%%%%%%%%%%%%%%%%%%%%%%%%
%

\newcounter{definitioncounter}
\setcounter{definitioncounter}{0}

\newcounter{theoremcounter}
\setcounter{theoremcounter}{0}

\newcounter{keyideacounter}
\setcounter{keyideacounter}{0}

\newlength{\specialboxlength}
\newlength{\specialboxtitlelength}
\newlength{\specialboxinnerseplength}

\setlength{\specialboxtitlelength}{75pt}
\setlength{\specialboxinnerseplength}{15pt}
\setlength{\specialboxlength}{\textwidth-\specialboxtitlelength-2\specialboxinnerseplength}

\definecolor{myblue}{rgb}{.7,.7,1}
\definecolor{mygreen}{rgb}{.7,1,.7}
\definecolor{myred}{rgb}{1,.7,.7}
\definecolor{myyellow}{rgb}{.9,.9,0}

\newtheoremstyle{mystyle}% Generic Theorem Style
  {15pt}%      Space above
  {15pt}%      Space below
  {\sffamily}%         Body font
  {}%         Indent amount (empty = no indent, \parindent = para indent)
  {\sffamily\bfseries}% Thm head font
  {}%        Punctuation after thm head
  {1.5em}%     Space after thm head: " " = normal interword space;
        %       \newline = linebreak
  {}%         Thm head spec (can be left empty, meaning `normal')

\newtheoremstyle{myexamplestyle}% Used for the Example environments
  {15pt}%      Space above
  {5pt}%      Space below
  {}%         Body font
  {}%         Indent amount (empty = no indent, \parindent = para indent)
  {\bfseries}% Thm head font
  {}%        Punctuation after thm head
  {1.5em}%     Space after thm head: " " = normal interword space;
        %       \newline = linebreak
  {}%         Thm head spec (can be left empty, meaning `normal')

\theoremstyle{mystyle}

\newcommand{\definition}[2]{%
\vskip\baselineskip
\noindent\refstepcounter{definitioncounter}\label{#1}%
\begin{tikzpicture}
\draw (0,0) node[anchor=north west,text width=80pt,inner sep=0pt]{\bf Definition \thedefinitioncounter};
\ifthenelse{\boolean{in_color}} %first, in color
{\draw (\specialboxtitlelength,0) node[rectangle,text width = \specialboxlength,very thick,inner sep=\specialboxinnerseplength,draw = yellow!95!black!60,top color = white!95!yellow,bottom color = yellow!90!black!30,text justified,anchor=north west,yshift=\specialboxinnerseplength] {#2};}%if not in color
{\draw (\specialboxtitlelength,0) node[rectangle,text width = \specialboxlength,very thick,inner sep=\specialboxinnerseplength,draw,text justified,anchor=north west,yshift=\specialboxinnerseplength] {#2};}
\end{tikzpicture}
}

%%%%%%%%%%%%%%%%%%%%%%%%%%%%%%%%%%%%%%%%%%%%%%%%%%%%%%%%%%%%%%%%%%%%%%%%%%%%%%%%%%%%%%%%
%%%%%%%%%%%%%%%%%%%%%%%%%%%%%%%%%%%%%%%%%%%%%%%%%%%%%%%%%%%%%%%%%%%%%%%%%%%%%%%%%%%%%%%%

\newcommand{\theorem}[2]{%
\vskip\baselineskip
\noindent\refstepcounter{theoremcounter}\label{#1}%
\begin{tikzpicture}
\draw (0,0) node[anchor=north west,text width=80pt,inner sep=0pt]{\bf Theorem \thetheoremcounter};
\ifthenelse{\boolean{in_color}} %first, in color
{\draw (\specialboxtitlelength,0) node[rectangle,text width = \specialboxlength,very thick,inner sep=\specialboxinnerseplength,draw = green!30!black!50,top color = white!95!green,bottom color = green!60!black!20,text justified,anchor=north west,yshift=\specialboxinnerseplength] {#2};}%if not in color
{\draw (\specialboxtitlelength,0) node[rectangle,text width = \specialboxlength,very thick,inner sep=\specialboxinnerseplength,draw,text justified,anchor=north west,yshift=\specialboxinnerseplength] {#2};}
\end{tikzpicture}
}


%%%%%%%%%%%%%%%%%%%%%%%%%%%%%%%%%%%%%%%%%%%%%%%%%%%%%%%%%%%%%%%%%%%%%%%%%%%%%%%%%%%%%%%%
%%%%%%%%%%%%%%%%%%%%%%%%%%%%%%%%%%%%%%%%%%%%%%%%%%%%%%%%%%%%%%%%%%%%%%%%%%%%%%%%%%%%%%%%

\newcommand{\idea}[2]{%
\vskip\baselineskip
\noindent\refstepcounter{keyideacounter}\label{#1}%
\begin{tikzpicture}
\draw (0,0) node[anchor=north west,text width=80pt,inner sep=0pt]{\bf Key Idea \thekeyideacounter};
\ifthenelse{\boolean{in_color}} %first, in color
{\draw (\specialboxtitlelength,0) node[rectangle,text width = \specialboxlength,very thick,inner sep=\specialboxinnerseplength,draw = red!30!black!50,top color = white!95!red,bottom color = red!60!black!20,text justified,anchor=north west,yshift=\specialboxinnerseplength] {#2};}%if not in color
{\draw (\specialboxtitlelength,0) node[rectangle,text width = \specialboxlength,very thick,inner sep=\specialboxinnerseplength,draw,text justified,anchor=north west,yshift=\specialboxinnerseplength] {#2};}
\end{tikzpicture}
}

%%%%
%% Begins the exercise section, containing all commands
%% related to creating problem sections.
%%%%

\newcommand{\exc}{\addtocounter{excounter}{2}\arabic{excounter}}

\newif\ifmore

\newif\ifexsetmore

% this counter gives an effective, albeit not elegant
% way of using the same command to both print questions
% or the answers, depending on what section you are in.
% showexercises = 1: print questions
% showexercises = 2: print odd answers only
% showexercises = 3: print all answers 
% 
% The subsequent lines sets the value 
\newcount\showexercises

\newcount\numberofexercises

\newcounter{numofexer}
\newcounter{negnumofexer}

% used for debugging; not really used anymore
\newcounter{debug}
\setcounter{debug}{0}

\newcounter{exercisecounter}
\newcounter{IMTcount}
\newcounter{IMTcount_temp}

% the exercise names can be printed next to the problem
% using these commands. The default is to not print them.
\newboolean{showexercisenames}
\newcommand{\printexercisenames}{\setboolean{showexercisenames}{true}}
\newcommand{\noprintexercisenames}{\setboolean{showexercisenames}{false}}
\noprintexercisenames


% TeX uses a certain system to name input files that it reads from.
% To prevent conflicts, we use the newread command.
\newread\exread %read an example
\newread\exsetread %read an example set
\newread\exansread %read the answer
\newread\printansread% read in the answers file

\newwrite\answrite %write the answers file
% give the answers file the name ``jobname.answers''
\openout\answrite=\jobname.answers

\def\exinput #1 {\ifnum \showexercises=1 
												\openin\exread=#1 
												\read\exread to \tempp 
												\begin{enumerate} 
													\addtocounter{enumi}{\theexercisecounter}
													\item 
													\ifthenelse{\boolean{showexercisenames}}
													{\tiny {\hskip -60pt}% This line too
												  \makebox[60pt][l]{\printexercisename #1 }%  
												  \small%
												  }{}
													\tempp 
													\addtocounter{exercisecounter}{1}
												\end{enumerate}
												\closein\exread 
									\else \ifnum \showexercises=2  %i.e, we are printing odd answers, not questions 
												\openin\exread=#1 
												\read\exread to \tempp % read in the question - we ignore it.
												\addtocounter{exercisecounter}{1}
												\read\exread to \tempp % reads in the answer
												\ifodd \theexercisecounter
												%\else
													\begin{enumerate} 
													\addtocounter{enumi}{\theexercisecounter}
													\addtocounter{enumi}{-1}
													\item 
													\ifthenelse{\boolean{showexercisenames}}
													{\tiny {\hskip -60pt}% This line too
												  \makebox[60pt][l]{\printexercisename #1 }%
												  \small%
												  }{}
													\tempp 
													%\addtocounter{exercisecounter}{1}
													\end{enumerate} 
												\fi
												\closein\exread 
												\fi  %ends the \ifnum \showexercises = 2 if statement
												%
												\ifnum \showexercises=3  %i.e., we print all answers, not just odds 
												\openin\exread=#1 
												\read\exread to \tempp %reads in the question, which is ignored 
												\read\exread to \tempp %reads in the answer
												\begin{enumerate} 
													\addtocounter{enumi}{\theexercisecounter}
													\item 
													\ifthenelse{\boolean{showexercisenames}}
													{\tiny {\hskip -60pt}% This line too
												  \makebox[60pt][l]{\printexercisename #1 }%
												  \small%
												  }{}
													\tempp 
													\addtocounter{exercisecounter}{1}
												\end{enumerate}
												\closein\exread
												\fi % ends the \ifnum \showexercises=3 if statement
									\fi 
								}

\def\exsetinput #1 {\openin\exsetread=#1
										\setcounter{numofexer}{0}
										\setcounter{negnumofexer}{0} 
										\read\exsetread to \exsettemp
										\read\exsetread to \exsettemp
										{\loop
												\read\exsetread to \exsettemp
												\ifeof \exsetread \exsetmorefalse \else \exsetmoretrue \fi
												\ifexsetmore
														\addtocounter{numofexer}{1}
														\addtocounter{negnumofexer}{-1}
											\repeat}							
										\closein\exsetread
										\openin\exsetread=#1
										\ifnum \showexercises=1 
											\read\exsetread to \exsettemp
											\setcounter{enumi}{\theexercisecounter} 
											\addtocounter{enumi}{1}
											\ifthenelse{\boolean{showexercisenames}}
													{\tiny {\hskip -60pt}% This line too
												  \makebox[60pt][l]{\printexercisename #1 }%
												  \small%
												  }{}
											\noin\textbf{\exsettemp\theenumi\addtocounter{enumi}{-1}
											\addtocounter{enumi}{\thenumofexer}{-- }\theenumi%
											\addtocounter{enumi}{\thenegnumofexer}%
											\read\exsetread to \exsettemp \exsettemp}%
											
											{\loop
													\read\exsetread to \exsettemp
													\ifeof \exsetread \exsetmorefalse \else \exsetmoretrue \fi
													\ifexsetmore
															\exsettemp
											\repeat}
										\else
											\read\exsetread to \exsettemp
											\read\exsetread to \exsettemp
											{\loop
													\read\exsetread to \exsettemp
													\ifeof \exsetread \exsetmorefalse \else \exsetmoretrue \fi
													\ifexsetmore
															\exsettemp
											\repeat}
										\fi
										\closein\exsetread
								}

\def\printexercises #1 {%
\ifthenelse{\equal{\expandafter\readsection\thesection}{1}}{\immediate\write\answrite{\noexpand \noindent {\noexpand\Large\noexpand\bf Chapter \thechapter} \noexpand \vskip \noexpand\baselineskip } \write\answrite{}}{}%
\immediate\write\answrite{\noexpand\noindent {\noexpand\bf Section \thesection} \noexpand \vskip \baselineskip}%
\write\answrite{\noexpand\printanswers{#1}}%
\setcounter{exercisecounter}{0}\showexercises=1 \small%
\noin\underline{\parbox{\textwidth}{\Large\textbf{Exercises \thesection} }}% 
\sffamily%
\vskip\baselineskip%
\begin{multicols}{2}%
				\openin\exansread=#1 
				\ifeof \exansread 
					{No problems written.} 
				\else 
					\loop \read\exansread to \extemp  
							\ifeof \exansread \morefalse \else \moretrue \fi 
							\ifmore 
									\extemp
							\repeat 
				\fi 
				\closein\exansread 
				\end{multicols}%
				\setlength{\hoffset}{0pt} \rmfamily\normalsize \vskip \baselineskip%
				}
				


% The following prints the answers. In order to print just odds, set \showexercises=2. 
% To print all answers, set \showexercises=3.
% 
\def\printanswers #1 {\setcounter{exercisecounter}{0} \footnotesize \showexercises=2 \openin\printansread=#1 
				\ifeof \printansread 
					{No problems written.} 
				\else 
					\loop \read\printansread to \extemp  
							\ifeof \printansread \morefalse \else \moretrue \fi 
							\ifmore 
									\extemp
							\fi 
							\ifeof \printansread \morefalse \else \moretrue \fi 
							\ifmore 
					\repeat 
				\fi 
				\closein\printansread
				\small}

\def\testexinput #1 {\ifnum \showexercises=1 
												\openin\exread=#1 
												\read\exread to \tempp 
												\begin{enumerate} 
													\addtocounter{enumi}{\theexercisecounter}
													\item \tempp 
													\addtocounter{exercisecounter}{1}
												\end{enumerate}
												\closein\exread 
									\else \ifnum \showexercises=2 
												\openin\exread=#1 
												\read\exread to \tempp 
												\addtocounter{exercisecounter}{1}
												\read\exread to \tempp 
												\ifodd \theexercisecounter
														\begin{enumerate} 
													\addtocounter{enumi}{\theexercisecounter}
													\addtocounter{enumi}{-1}
													\item \tempp 
													%\addtocounter{exercisecounter}{1}
													\end{enumerate} 
												\fi
												\closein\exread 
												\fi
									\fi 
								}
								
\def \printexercisename exercises/#1_#2_#3_#4 {#1 #2 #3 #4}

%%%%%%%%%%%%% Used to automate the answer production at
%%%%%%%%%%%%% end the text.

\def \readsection #1.#2{#2}

\def \writeexercisestofile #1{%
\write\answrite{\noexpand\printanswers{exercises/0\thechapter_0\expandafter\readsection #1_exercises.tex} \noexpand \vskip \baselineskip } \write\answrite{} }

\def \makeexercisesection #1{\write\answrite{\noexpand\end{multicols}
\noexpand\normalsize} \closeout\answrite \chapter{#1} \input{\jobname.answers}}%

%
% The following is a line of code, not a definition. 
% It writes the first line of the ``.answers'' file
% to set up the proper formatting of that file.
%
\write\answrite{\noexpand\small\noexpand\raggedright\noexpand\begin{multicols}{2}}

\printincolor