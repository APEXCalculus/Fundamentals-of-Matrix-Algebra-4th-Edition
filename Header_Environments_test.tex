\newlength{\topmarginlength}
\newlength{\bottommarginlength}
\newlength{\oddpagemarginlength}
\newlength{\evenpagemarginlength}
\newlength{\marginlinelength}

\newcounter{examplestartpageref}
\newcounter{pagedifference}

\setlength{\marginlinelength}{5pt}
\setlength{\topmarginlength}{-1in-\voffset}
\setlength{\bottommarginlength}{-1in-\textheight-2\baselineskip}
\setlength{\oddpagemarginlength}{1in+\hoffset+\oddsidemargin-2\marginlinelength}
\setlength{\evenpagemarginlength}{1in+\hoffset+\evensidemargin-2\marginlinelength}

\newcommand{\colorlinecolor}{blue!95!black!30}
\newcommand{\bwlinecolor}{black!30}
\ifthenelse{\boolean{in_color}}{\newcommand{\thelinecolor}{\colorlinecolor}}{\newcommand{\thelinecolor}{\bwlinecolor}}
\newcommand{\linestyle}{thick}

\newcounter{examplecounter}
\setcounter{examplecounter}{0}

\newcommand{\example}[3]{%
\noindent%
\hskip -2\marginlinelength%
\parbox{\marginlinelength}{%
\begin{tikzpicture}[remember picture,overlay]%
\draw (0,0) circle (1pt) node (#1) {};%
\end{tikzpicture}%
}% ends parbox where line begins
\hskip \marginlinelength% puts us back in line
\parbox{80pt}{{\bf Example \refstepcounter{examplecounter}\theexamplecounter}}% ends parbox
\label{#1}%
\ifthenelse{\pageref{#1}=\pageref{e#1}}% if the beginning and end are on the same page
{%
#2 \par #3 {\it Selah}\label{e#1}% writes the full example then draws the lines
\begin{tikzpicture}[remember picture,overlay] \draw (0,0) circle (1pt) node (e#1) {};
\draw [\linestyle,\thelinecolor] (#1) -- (#1.south |- e#1.south) -- ++(10pt,0);
\end{tikzpicture}
}% ends if beginning and end are on same page.
% next is if they are on different pages
{% first draw line from start to bottom of page
\begin{tikzpicture}[remember picture,overlay]% 
\ifthenelse{\isodd{\pageref{#1}}}% draws lines based on whether on an even or odd page %
{\node [xshift=\oddpagemarginlength,yshift=\bottommarginlength](bottomleft) at (current page.north west)  {};}
{\node [xshift=\evenpagemarginlength,yshift=\bottommarginlength](bottomleft) at (current page.north west)  {};}
\draw [green] (#1) -- (bottomleft);
\end{tikzpicture}
% end drawing of line
#2 \par #3 {\it Selah}\label{e#1}% now write out full example
% now draw line from end to top of page
\begin{tikzpicture}[remember picture,overlay] \draw (0,0) circle (1pt) node (e#1) {};
\ifthenelse{\isodd{\pageref{e#1}}}% draws lines based on whether on an even or odd page %
{\node [xshift=\oddpagemarginlength,yshift=\topmarginlength](topleft) at (current page.north west)  {};}
{\node [xshift=\evenpagemarginlength,yshift=\topmarginlength](topleft) at (current page.north west)  {};}
\draw [blue,very thick] (e#1.south -| topleft) -- (topleft);
\end{tikzpicture}%
}% ends the check for same page or not
%
}%ends the definition of exam


% Draws a line on a page that doesn't contain either
% the beginning or end of an example.
% It has one arguments, the  label
% of the current example. 
\newcommand{\drawexampleline}[1]{
\ifthenelse{\boolean{showexamplelines}}
{ % show lines
\setcounterpageref{examplestartpageref}{#1}%
\setcounterpageref{pagedifference}{\thepage-\theexamplestartpageref}
\ifthenelse{\isodd{\theexamplestartpageref}}% example is on an odd page
       {\ifthenelse{\isodd{\thepagedifference}}% is the difference between this page
       % and the start even or odd?
       % odd: then plots on different side of page
       {\begin{tikzpicture}[remember picture,overlay]
        \draw [xshift=\evenpagemarginlength,yshift=\topmarginlength,\thelinecolor] (current page.north west) -- +(0,-\textheight);
        \end{tikzpicture}
        }% ends if difference is odd
        {% difference is even
        \begin{tikzpicture}[remember picture,overlay]
        \draw [xshift=\oddpagemarginlength,yshift=\topmarginlength,\thelinecolor] (current page.north west) -- +(0,-\textheight);
        \end{tikzpicture}
        }% ends difference is even and start page is odd
}% ends start page is odd
       {\ifthenelse{\isodd{\thepagedifference}}% is the difference between this page
       % and the start even or odd?
       % odd: then plots on different side of page
       {\begin{tikzpicture}[remember picture,overlay]
        \draw [xshift=\oddpagemarginlength,yshift=\topmarginlength,\thelinecolor] (current page.north west) -- +(0,-\textheight);
        \end{tikzpicture}
        }% ends if difference is odd
        {% difference is even
        \begin{tikzpicture}[remember picture,overlay]
        \draw [xshift=\evenpagemarginlength,yshift=\topmarginlength,\thelinecolor] (current page.north west) -- +(0,-\textheight);
        \end{tikzpicture}
        }% ends difference is even and start page is even
}% ends start page is even
} % End show lines
{} % No else - don't do anything
}



%
%Define style for Definitions, Theorems and Key Ideas  %%%%%%%%%%%%%%%%%%%%%%%%%%%%%%%%%%%%%%%%%%%%
%%%%%%%%%%%%%%%%%%%%%%%%%%%%%%%%%%%%%%%%%%%%%%%%%%%%%%%%%%%%%%%%%%%%%%%%%%%%%%%%%%%%%%%%%%%%%%%%%%%
%

\theoremstyle{mystyle}

\newtheorem{deff}{Definition}

%\newcommand{\definition}[2]{\begin{deff}\label{#1} \fcolorbox{blue}{myblue}{ \begin{minipage}[t]{265pt}#2\end{minipage}}\end{deff}}

\ifthenelse{\boolean{booksize}\or\boolean{amazonsize}}{% Begin if booksize
\newcommand{\definition}[2]{\begin{deff}\label{#1}
\raisebox{24pt}{\begin{tikzpicture}[baseline=(current bounding box.north)-1cm]
%\pgfsetbaselinepointlater{\pgfpointanchor{X}{base}}
\ifthenelse{\boolean{in_color}}{\node (X) [rectangle,
text width = 240pt,
baseline=-1cm,
very thick,
inner sep=15pt,
draw = yellow!95!black!60,
top color = white!95!yellow,
bottom color = yellow!90!black!30,
text justified]} 
{\node (X) [rectangle,draw,
text width = 240pt,
baseline=-1cm,
inner sep=15pt,
very thick,
text justified]} 
{#2};
\end{tikzpicture}}
\end{deff}}%
}%  End If
{% begin else not booksize
\newcommand{\definition}[2]{\begin{deff}\label{#1}
\raisebox{24pt}{\begin{tikzpicture}[baseline=(current bounding box.north)-1cm]
%\pgfsetbaselinepointlater{\pgfpointanchor{X}{base}}
\node (X) [rectangle,draw,
text width = 170pt,
baseline=-1cm,
inner sep=15pt,
very thick,
text justified] 
{#2};
\end{tikzpicture}}
\end{deff}}%
}%

%%%%%%%%%%%%%%%%%%%%%%%%%%%%%%%%%%%%%%%%%%%%%%%%%%%%%%%%%%%%%%%%%%%%%%%%%%%%%%%%%%%%%%%%
%%%%%%%%%%%%%%%%%%%%%%%%%%%%%%%%%%%%%%%%%%%%%%%%%%%%%%%%%%%%%%%%%%%%%%%%%%%%%%%%%%%%%%%%

\newtheorem{thm}{Theorem}

\ifthenelse{\boolean{booksize}\or\boolean{amazonsize}}{%
\newcommand{\colortheorem}[2]{\begin{thm}\label{#1}
\raisebox{24pt}{\begin{tikzpicture}[baseline=(current bounding box.north)-1cm]
%\pgfsetbaselinepointlater{\pgfpointanchor{X}{base}}
\ifthenelse{\boolean{in_color}}
{\node (X) [rectangle,
text width = 240pt,
baseline=-1cm,
inner sep=15pt,
very thick,
draw = green!30!black!50,
top color = white!95!green,
bottom color = green!60!black!20,
text justified] }
{\node (X) [rectangle,draw,
text width = 240pt,
baseline=-1cm,
inner sep=15pt,
very thick,
text justified] }
{#2};
\end{tikzpicture}}\end{thm}}%
}%
{%
\newcommand{\colortheorem}[2]{\begin{thm}\label{#1}
\raisebox{24pt}{\begin{tikzpicture}[baseline=(current bounding box.north)-1cm]
%\pgfsetbaselinepointlater{\pgfpointanchor{X}{base}}
\node (X) [rectangle,draw,
text width = 170pt,
baseline=-1cm,
inner sep=15pt,
very thick,
text justified] 
{#2};
\end{tikzpicture}}\end{thm}}%
}%

%%%%%%%%%%%%%%%%%%%%%%%%%%%%%%%%%%%%%%%%%%%%%%%%%%%%%%%%%%%%%%%%%%%%%%%%%%%%%%%%%%%%%%%%
%%%%%%%%%%%%%%%%%%%%%%%%%%%%%%%%%%%%%%%%%%%%%%%%%%%%%%%%%%%%%%%%%%%%%%%%%%%%%%%%%%%%%%%%

\newtheorem{keyidea}{Key Idea}

\ifthenelse{\boolean{booksize}\or\boolean{amazonsize}}{%
\newcommand{\idea}[2]{\begin{keyidea}\label{#1}
\raisebox{24pt}{\begin{tikzpicture}[baseline=(current bounding box.north)-1cm]
%\pgfsetbaselinepointlater{\pgfpointanchor{X}{base}}
\ifthenelse{\boolean{in_color}}
{\node (X) [rectangle,
text width = 240pt,
baseline=-1cm,
inner sep=15pt,
very thick,
draw = red!30!black!50,
top color = white!95!red,
bottom color = red!60!black!20,
text justified] }
{\node (X) [rectangle,draw,
text width = 240pt,
baseline=-1cm,
inner sep=15pt,
very thick,
text justified] }
{#2};
\end{tikzpicture}}
\end{keyidea}}%
}%
{%
\newcommand{\idea}[2]{\begin{keyidea}\label{#1}
\raisebox{24pt}{\begin{tikzpicture}[baseline=(current bounding box.north)-1cm]
%\pgfsetbaselinepointlater{\pgfpointanchor{X}{base}}
\node (X) [rectangle,draw,
text width = 170pt,
baseline=-1cm,
inner sep=15pt,
very thick,
text justified] 
{#2};
\end{tikzpicture}}
\end{keyidea}}%
}%

%%%%%%%%%%%%%%%%%%%%%%%%%%%%%%%%%%%%%%%%%%%%%%%%%%%%%%%%%%%%%%%%%%%%%%%%%%%%%%%%%%%%%%%%
%%%%%%%%%%%%%%%%%%%%%%%%%%%%%%%%%%%%%%%%%%%%%%%%%%%%%%%%%%%%%%%%%%%%%%%%%%%%%%%%%%%%%%%%

\newcommand{\asyouread}[1]{\begin{tikzpicture}
\ifthenelse{\boolean{in_color}}{\node [preaction={fill=black,opacity=.5,transform canvas={xshift=1mm,yshift=-1mm}}, right color=blue!80!black!30, left color=blue!80] at (0,0) {\textcolor{white}{\textsf{\textit{AS YOU READ $\ldots$}}}};}
{\node [preaction={fill=black,opacity=.5,transform canvas={xshift=1mm,yshift=-1mm}}, right color=black!30, left color=black!10] at (0,0) {\textcolor{white}{\textsf{\textit{AS YOU READ $\ldots$}}}};}
\end{tikzpicture}
\begin{enumerate}
#1
\end{enumerate}
\vskip 20pt}
